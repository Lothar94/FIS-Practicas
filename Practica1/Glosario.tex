\documentclass{article}
\usepackage[utf8]{inputenc}
\usepackage{enumerate}
\usepackage{enumitem}
\usepackage{float}
\usepackage{graphicx}
\usepackage{multirow, array}


\title{Práctica 1: Glosario de términos}
\author{Rafael Nogales Vaquero
\\Lothar Soto Palma
\\Elena Toro Pérez
\\Jose Ramón Trillo Vilchez}
\date{15 Octubre del 2014}

\begin{document}

\maketitle

\begin{description}
\item[Sistema:] En esta ocasion se trata de una web de películas de cine.
\item[Usuario:] Persona que usa habitualmente el sistema y sus servicios.
\item[Cliente:] Persona no registrada en el sistema.
\item[Administrador:] Persona que gestiona la web en cuestión.
\item[Empleado:] Persona que junto a un director o administrador se encarga del mantenimiento de los servicios de la web.
\item[Usuario sistema:] Usuario que hace uso del sistema.
\item[Usuario producto:] Usuario con privilegios que se encarga del sistema.
\item[Registro:] Proceso de asignación de una identificación para un cliente.
\item[Nick:] Nombre que se le dá al usuario.
\item[Implicados:] Conjunto de usuarios, clientes, empleados y administradores.
\item[Nota media:] Valoración de una película.
\item[Banear:] Bloquear total o parcialmente el acceso a un usuario conflictivo.
\item[Algoritmo KNN:] Se encarga de la asociación de almas gemelas entre usuarios de manera que todos posean una lista de amigos con gustos cercanos.
\item[Interfaz:] Aquello que permite a los implicados hacer uso del sistema y de sus servicios.
\item[Incidencia:] Problema ocasionado en la web que puede variar dependiendo de su gravedad.
\item[Datos:] Conjunto de información acerca de los implicados, las películas y el sistema en general.
\item[Lenguaje de Programación:] Aquello con lo que se construirá el sistema.

\end{description}

\end{document}
