\documentclass{article}
\usepackage[utf8]{inputenc}

\title{Gestión de una Web de cine}
\author{Rafael Nogales Vaquero
\\Lothar Soto Palma
\\Elena Toro Pérez
\\Jose Ramón Trillo Vilchez}
\date{15 Octubre del 2014}

\begin{document}

\maketitle

\section{Objetivos}

Este proyecto tiene como principal objetivo desarrollar un entorno web para gente con gran interés en el mundo del cine.
Estos aficionados podrán consultar on-line, en todo momento, las puntuaciones sobre cualquier película, información
detallada sobre estrenos y la posibilidad de comprar cualquier DVD que esté en el mercado.
También podrán saber las puntuaciones que sus amigos o conocidos les han dado a ciertas películas, podrán acceder a
recomendaciones por parte del conjunto de los usuarios y consultar los estrenos de la cartelera actuales o próximos a
extrenarse.
Además, cada vez que busquen una película en el buscador, podrán acceder a las puntuaciones de dicha película por parte de
sus amigos o conocidos más cercanos.
En el caso de no estar registrado y tener la opción de agregar amigos, le mostraremos la puntuación media de la película
por parte de todos los internautas que la han puntuado.
Otras de las grandes ventajas de nuestra web es que podrá escribir críticas sobre cualquier película y darle la puntuación
conveniente. Además de crearse listas de películas (las favoritas, las que quiere ver, etc.).
\section{Descripción de los implicados}
\section{Requisitos}
    \subsection{Funcionales}
    \subsection{No Funcionales}
    \subsection{Informacion}



%EJEMPLO DE TABLA A USAR EN LATEX

\begin{tabular}{||l | l | l | l ||}
\hline
\hline
columna 1 & columna 2 & columna 3 & columna 4\\
\hline
col 1 & col 2 & col 3 & 4\\
\hline
1 & 2 & 3 & 4\\
\hline
\hline
\end{tabular}


\end{document}