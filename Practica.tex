\documentclass{article}
\usepackage[utf8]{inputenc}
\usepackage{enumerate}
\usepackage{enumitem}


\title{Gestión de una Web de cine}
\author{Rafael Nogales Vaquero
\\Lothar Soto Palma
\\Elena Toro Pérez
\\Jose Ramón Trillo Vilchez}
\date{15 Octubre del 2014}

\begin{document}

\maketitle

\section{Objetivos}
Este proyecto tiene como principal objetivo desarrollar un entorno web para gente con gran interés en el mundo del cine.
Estos aficionados podrán consultar on-line, en todo momento, las puntuaciones sobre cualquier película, información
detallada sobre estrenos y la posibilidad de comprar cualquier DVD que esté en el mercado.
También podrán saber las puntuaciones que sus amigos o conocidos les han dado a ciertas películas, podrán acceder a
recomendaciones por parte del conjunto de los usuarios y consultar los estrenos de la cartelera actuales o próximos a
extrenarse.
Además, cada vez que busquen una película en el buscador, podrán acceder a las puntuaciones de dicha película por parte de
sus amigos o conocidos más cercanos.
En el caso de no estar registrado y tener la opción de agregar amigos, le mostraremos la puntuación media de la película
por parte de todos los internautas que la han puntuado.
Otras de las grandes ventajas de nuestra web es que podrá escribir críticas sobre cualquier película y darle la puntuación
conveniente. Además de crearse listas de películas (las favoritas, las que quiere ver, etc.).
\section{Descripción de los implicados}
\section{Requisitos}
    \subsection{Funcionales}
    \subsection{No Funcionales}
    \begin{itemize}
    \item Usabilidad
        \begin{enumerate}[label=\bfseries RN- \arabic*:]
        \item El sistema está pensado para personas que no tienen experiencia puesto que la interfaz de la misma es bastante sencilla.
        \item Se indicará al usuario en portada acerca del conjunto de peliculas que se encuentra en cartelera actualmente además de las recomendaciones correspondientes y peliculas que saldrán proximamente en la cartelera.
        \item El sistema tal y como esta diseñado dará instrucciones acerca de lo que el usuario es capaz de hacer en cada momento.
        \item Se proporcionará ayudas al usuario, el usuario puede avisar de errores en la información proporcionada por las peliculas, o bien sean errores que se han producido en la web, el empleado correspondiente se encargará del análisis de dichos avisos. 
        \end{enumerate}
    \item Fiabilidad
        \begin{enumerate}[label=\bfseries RN- \arabic*:]
        \setcounter{enumi}{4}
        \item El sistema deberá realizar copias de seguridad cada cierto tiempo, debido a los posibles errores o caidas del sistema.
        \item El sistema proporcionará un aviso de la incidendia al usuario en caso de no poder volver a restaurar el sistema.
        \item Se deberán de tener varios servidores interconectados atendiendo a los usuarios, en caso de error en alguno de los servidores la carga de usuarios se dividirá en el resto de servidores funcionales.
        \item El sistema deberá asegurar el estado de los datos una vez recupere un buen estado de funcionalidad.
        \item El sistema estará regido por una jerarquia de empleados que se encargaran de la seguridad de la misma, además de esto a los usuarios se les pedirá información relevante para facilitar su identificación y evitar cuentas de robots automaticos. 
        \end{enumerate}
    \item Rendimiento
        \begin{enumerate}[label=\bfseries RN- \arabic*:]
        \setcounter{enumi}{9}
        \item Se separarán las capacidades de un usuario de la compra de DVD/BlueRay, es decir, se usará una interfaz de compra de películas diferente al entorno usual del usuario.
        \item La base de datos tendrá que estar preparada para alojar tantos usuarios como sea necesario, alrededor de 100000 usuarios registrados en un inicio más esto se deberá incrementar posterirormente.
        \item Cuando el usuario compra una película no deberá encontrarse ningun estado de indeterminación donde este no sepa la consecuencia de su acción.
        \end{enumerate}
    \item Soporte
        \begin{enumerate}[label=\bfseries RN- \arabic*:]
        \setcounter{enumi}{12}
        \item El sistema se encontrará mantenido por un grupo de empleados organizados de manera jerarquica y repartidos por secciones, estos se encontrarán dirigidos por un director de proyecto o administrador del sistema.
        \item Será posible implementar la web para que tenga soporte en dispositivos moviles y tablets del mercado.
        \end{enumerate}
    \item Restricciones de implementeción
        \begin{enumerate}[label=\bfseries RN- \arabic*:]
        \setcounter{enumi}{14}
        \item El sistema puesto al ser una web necesitará de los lenguajes siguientes:
            \begin{itemize}
            \item HTML5
            \item CSS3
            \item PHP
            \item MySQL
            \end{itemize}
        \item El sistema también podrá ser implementado por la siguiente combinación:
            \begin{itemize}
            \item HTML5
            \item CSS3
            \item Ruby on Rails
            \item MySQL o SQLite3
            \end{itemize}
        \end{enumerate}
    \item Requisitos físicos
        \begin{enumerate}[label=\bfseries RN- \arabic*:]
        \setcounter{enumi}{16}
        \item El sistema deberá tratar los datos de usuario en tiempo real, además de adaptar las peliculas para cada uno de los usuarios, y proporcionarles un metodo cómodo de compra.
        \item Los datos del sistema de base de datos usados estará actualizado en todo momento por posibles peticiones a la misma.
        \end{enumerate}
    \item Operación
        \begin{enumerate}[label=\bfseries RN- \arabic*:]
        \setcounter{enumi}{18}
        \item El sistema estará operado por un equipo de empleados mientres este ofrezca servicio.
        \end{enumerate}
    \item Legales
        \begin{enumerate}[label=\bfseries RN- \arabic*:]
        \setcounter{enumi}{19}
        \item El sistema estará protegido por derechos de autor y Copyright.
        \item El sistema no proporcionará datos de los usuarios a ninguna otra entidad.
        \item El sistema ofrece una compra segura y dentro de la legalidad vigente con periodos de reclamación correspondiente.
        \end{enumerate}
    \end{itemize}
    \subsection{Informacion}



%EJEMPLO DE TABLA A USAR EN LATEX

\begin{tabular}{||l | l | l | l ||}
\hline
\hline
columna 1 & columna 2 & columna 3 & columna 4\\
\hline
col 1 & col 2 & col 3 & 4\\
\hline
1 & 2 & 3 & 4\\
\hline
\hline
\end{tabular}


\end{document}
